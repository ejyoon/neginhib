% neginhib cogsci submission


\documentclass[10pt,letterpaper]{article}

\usepackage{cogsci}
\usepackage{pslatex}
\usepackage{apacite}


\title{inhibition, negation and implicature processing}
 
\author{{\large \bf Ann Nordmeyer} \\
  \texttt{anordmey@stanford.edu} \\
  Department of Psychology \\
  Stanford University
  \And {\large \bf Erica J. Yoon} \\
  \texttt{ejyoon@stanford.edu} \\
  Department of Psychology \\
  Stanford University
  \And {\large \bf Michael C. Frank} \\
  \texttt{mcfrank@stanford.edu} \\
  Department of Psychology \\
  Stanford University}

\begin{document}

\maketitle


\begin{abstract}

...

\textbf{Keywords:} 
Inhibitory control; negation; implicature; drift diffusion model; cognitive development; pragmatics

\end{abstract}


\section{Introduction}

Previous research has shown that children have a hard time processing negation \cite{nordmeyer2014role} and implicature \cite{yoonchildren}. One possible reason is inhibitory demand of these processes, rather than the lack of pragmatic understanding, which children demonstrate from an early age (e.g., \citeNP{katsos2011pragmatic, matthews2012two}).

The current study uses a simple game with three phases that test children's processing of inhibitory demands, negation and implicature, respectively. We used nearly identical auditory and visual stimuli for the three phases, which allowed direct comparisons. % of... 

We use the drift diffusion model \cite{ratcliff1978theory} to look at the deeper meanings of behavioral responses that may look similar on surface. We also compare our data with a previous diffusional model analysis of developmental changes in a different cognitive task \cite{ratcliff2012children} and show our results align.

We also compared performances on tablet vs. computer, which will be informative for future studies that use these tools.

\section{Method}

\subsection{Participants}

We invited children at Bing Nursery School at Stanford University and Children'��s Discovery Museum in San Jose, CA, to play a game to find things that are named. 4-year-olds (N = FIXME; M = FIXME) played the game on an iPad, and 4-, 5-, and 6-year-olds played the game on a computer (N = FIXME respectively; M = FIXME respectively). (FIXME: exclusion criteria?) We also recruited adult participants (N = FIXME) on Amazon Mechanical Turk to play the computer version of the task.  

\subsection{Stimuli and Design}

The game consisted of three phases that each tested one of the target processes: inhibition, negation processing, and implicature processing. In each trial in all three phases, there were two images side-by-side on the screen. a Pre-recorded voice said one or two words to refer to the target image, and participants' task was to select the correct referent as soon as they could identify it.

For the ``inhibition'' phase, in a set of 6-8 trials, the same two pictures appeared together (e.g., a picture of a carrot and a picture of a banana), with randomized sides. For the first 5-7 trials ({\emph control}), the same object was named (e.g., ``carrot''), then on the last trial  ({\emph inhibition}), the other object was named (``banana''). Participants were predicted to gain speed on every trial until the last one, in which the accuracy rate was expected to fall, due to the inhibitory demand. Child participants saw a total of 12 sets of trials, and adults saw 24 sets, in the inhibition phase.

For the ``negation'' phase, the referents were named with or without negation. For example, given two pictures of carrot and banana respectively, to refer to the banana the recorded voice said ``banana'' (in a {\emph positive} trial) or ``no carrot'' (in a {\emph negative} trial). Children saw 60 trials, and adults saw 120 trials in the negation phase. 

For the ``implicature'' phase, in each trial there was a picture with one object (e.g., carrot) and another picture with the same object and another one (e.g., carrot and banana). In an {\emph unambiguous} trial, the unique object was named (``banana'') and in an {\emph implicature} trial, the common object was named (``carrot''), implying ``carrot {\emph but not banana''}. Children saw 60 trials, and adults saw 120 trials in the implicature phase. 

\subsection{Procedure}

Adult participants first went through two practice trials, where they were asked to select an obvious, unambiguous referent (e.g., ``cow'' as opposed to ``rabbit''). Then they went through the three test phases in a randomized order.

For child participants, an experimenter first explained the game. Then, for those participants playing the computer version, it proceeded in the same way as for adult participants. For participants playing the iPad version, they played a dots game, where they tapped five dots in random locations, which helped them practice using the iPad screen. Once participants were used to using the screen, they proceeded to the test phases. After the child participants were finished, the experimenter gave them a sticker as a gift and thanked them for playing the game.

\section{Results}

\section{Model}

\section{Discussion}

%\section{Acknowledgments}
%
%Bing Nursery School, CDM, Stephen Powell, Veronica, Rachel

\bibliographystyle{apacite}

\setlength{\bibleftmargin}{.125in}
\setlength{\bibindent}{-\bibleftmargin}

\bibliography{neginhib}


\end{document}
